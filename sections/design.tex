The flow query language consists of a number of independent stages that are
connected to one another to form a processing pipeline as shown in Fig.
\ref{fig:nfql-pipeline}. A complete description of the semantics of each
pipeline stage can be found in \cite{vmarinov:2009}. The pipeline starts with
the splitter. It takes the flow-records data as input in \texttt{flow-tools}
compatible format. The splitter is responsible to duplicate the input data out
to several branches without any processing whatsoever. This allows each of the
branches to receive an identical copy of the flow data to process it
independently.

The filter is the first processing stage of a branch in the pipeline.  It
performs \emph{absolute} filtering on the input flow-records data.  The
flow-records that pass the filtering criterion are forwarded to the grouper,
the rest of the flow-records are dropped. The filter compares separate fields
of a flow-record against either a constant value or a value on a different
field of the \emph{same} flow-record. The grouper performs aggregation of the
input flow-records. The grouping terms can be either absolute or relative. The
newly formed groups, which are passed on to the group filter, can also contain
meta-information about the flow-records contained within the group using the
aggregate clause defined as part of the grouper query. The possible
aggregation operations are \emph{static, count, product, sum, logical
and/or/xor, arithmetic mean, standard deviation, union, median, minimum and
maximum}. The group-filter is the last processing stage of a branch. It
performs \emph{absolute} filtering on group-records. The group-records that
pass the filtering criterion are forwarded to the merger, the rest of the
group-records are dropped. The group-filter compares separate fields (or
aggregated fields) of a group-record against either a constant value or a
value on a different field of the \emph{same} group-record. The merger
performs relative filtering on the N-tuples of groups formed from the N stream
of groups passed on from the group-filter as input. The relative filtering on
the groups are applied to express timing and concurrency constraints using
Allen interval algebra \cite{fallen:1983}. The ungrouper is the final
processing stage. It unwraps the tuples of group-records into individual
flow-records, ordered by their timestamps. Any duplicate flow-records
appearing from several group-records are eliminated.
