The flow query language pipeline consists of a number of independent elements
that are connected to one another using UNIX-based pipes.  Each element
receives the content from the previous pipe, performs an operation and pushes
it to the next element in the pipeline as shown in Fig.
\ref{fig:nfql-pipeline}. A complete description on the semantics of each
element in the pipeline can be found in \cite{vmarinov:2009}

The pipeline starts with the splitter. It takes the flow-records data as input
in the \texttt{flow-tools} compatible format. The splitter is responsible to
duplicate the input data out to several branches without any processing
whatsoever. This allows each of the branch to receive an identical copy of the
flow data to process it independently.

The filter is the first processing element of a branch in the pipeline.  It
performs \emph{absolute} filtering on the input flow-records data.  The
flow-records that pass the filtering criterion are forwarded to the grouper,
the rest of the flow-records are dropped. The filter compares separate fields
of a flow-record against either a constant value or a value on a different
field of the \emph{same} flow-record. The filter cannot \emph{relatively}
compare two different incoming flow-records

The grouper performs aggregation of the input flow-records data. It consists
of a number of terms that correspond to a specific subgroup. A flow-record in
order to be a part of the group should satisfy at-least one subgroup. A
flow-record can be a part of multiple subgroups within a group. A flow-record
cannot be part of multiple groups.  The grouping rules can be either absolute
or relative. The newly formed groups which are passed on to the group filter
can also contain meta-information about the flow-records contained within the
group using the aggregate clause defined as part of the grouper query.

The group-filter is the last processing element of the branch. It performs
\emph{absolute} filtering on the input group-records data. The group-records
that pass the filtering criterion are forwarded to the merger, the rest of the
group-records are dropped. The group-filter compares separate fields (or
aggregated fields) of a flow-record against either a constant value or a value
on a different field of the \emph{same} flow-record. The group-filter cannot
\emph{relatively} compare two different incoming group-records.

The merger performs relative filtering on the N-tuples of groups formed from
the N stream of groups passed on from the group-filter as input. The relative
filtering on the groups are applied to express timing and concurrency
constraints using Allen interval algebra \cite{fallen:1983}. The ungrouper is
the final processing element of the pipeline. It unwraps the tuples of
group-records into individual flow-records, ordered by their timestamps. The
duplicate flow-records appearing from several group-records are eliminated and
are sent as output only once.
