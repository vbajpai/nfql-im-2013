The pipeline consists of a number of independent processing elements that are
connected to one another using UNIX-based pipes. Each element receives the
content from the previous pipe, performs an operation and pushes it to the
next element in the pipeline. Fig. \ref{fig:nfql-pipeline} shows an
overview of the processing pipeline. A complete description on the semantics
of each element in the pipeline can be found in \cite{vmarinov:2009}

The splitter takes the flow-records data as input in the \texttt{flow-tools}
compatible format. It is responsible to duplicate the input data out to
several branches without any processing whatsoever. This
allows each of the branches to have an identical copy of the flow data to
process it independently.

The filter performs \emph{absolute} filtering on the input flow-records data.
The flow-records that pass the filtering criterion are forwarded to the
grouper, the rest of the flow-records are dropped. The
filter compares separate fields of a flow-record against either a constant
value or a value on a different field of the \emph{same} flow-record. The
filter cannot \emph{relatively} compare two different incoming flow-records

The grouper performs aggregation of the input flow-records data. It consists
of a number of rule modules that correspond to a specific subgroup. A
flow-record in order to be a part of the group should be a part of at-least
one subgroup. A flow-record can be a part of multiple subgroups within a
group. A flow-record cannot be part of multiple groups.
The grouping rules can be either absolute or relative. The newly formed groups
which are passed on to the group filter can also contain meta-information
about the flow-records contained within the group using the aggregate clause
defined as part of the grouper query.

The group-filter performs \emph{absolute} filtering on the input group-records
data. The group-records that pass the filtering criterion are forwarded to the
merger, the rest of the group-records are dropped. The group-filter compares
separate fields (or aggregated fields) of a
flow-record against either a constant value or a value on a different field of
the \emph{same} flow-record. The group-filter cannot \emph{relatively} compare
two different incoming group-records

The merger performs relative filtering on the N-tuples of groups formed from
the N stream of groups passed on from the group-filter as input. The merger
rule module consists of a number of a submodules, where the output of the
merger is the set difference of the output of the first
submodule with the union of the output of the rest of the submodules. The
relative filtering on the groups are applied to express timing and concurrency
constraints using Allen interval algebra \cite{fallen:1983}

The ungrouper unwraps the tuples of group-records into individual
flow-records, ordered by their timestamps. The duplicate
flow-records appearing from several group-records are eliminated and are sent
as output only once.

