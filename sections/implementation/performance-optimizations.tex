There can be a situation where the query designer may incorrectly ask for
aggregation on a field already specified in a grouper (or filter) clause. If
the relative operator is an equality comparison, the aggregation on such a
field becomes less useful, since the members of the grouped record will always
have the same value for that field. \ac{NFQL} realizes this redundant request
and ignores such aggregations.

\ac{NFQL} has dedicated comparator functions for each type of operation and
the type of the field offset it operates upon. It is not guaranteed that given
the type of the query and trace, the program will eventually go through each
stage of the pipeline. It is also possible that the program exits before,
because there is nothing more for the next stage to compute. The function
pointers are therefore set as late as possible and are are called from their
respective stages just before the comparison is needed.  As a result, we save
the computation time wasted in setting the function pointer for stage if it is
is never executed.

\ac{NFQL} pushes the filter stage out of the branch and closer to where the
trace is originally read. This reduces the memory allocation to only records
that passed the filter stage thereby reducing the memory footprint of the
execution engine.

Each stage of the processing pipeline is dependent on the result of the
previous one. As a result, the stages should only proceed, when the previous
returned results. Implementing such a response was straightforward for the
grouper and group filter, the merger although was a little trickier.  The
merger stage proceeds only when every branch has non-zero filtered groups.
The iterator initializer \texttt{iter\_init(\ldots)} deallocates and returns
\texttt{NULL} if any one branch has $0$ filtered groups.  Consequently a check
is performed in the merger to make sure \texttt{iter} is \emph{not}
\texttt{NULL}.

The results from each stage of the pipeline are echoed to the standard output
just before the execution engine exits. This leads to an additional loop to
echo the results, however echoes to the standard output are only used for
debugging purposes. The writes to a file can however be legitimately requested
in a real network analysis task. As a result, it is essential to avoid
additional loops when performing writes to a file. The execution engine,
therefore writes each result record to a file as soon as it is seen by the
pipeline stage.
