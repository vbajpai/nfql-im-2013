The \ac{NFQL} execution engine reads the flow query at runtime in a
\texttt{JSON} intermediate format. The entire flow query is a collection of
\ac{DNF} expression. A \ac{DNF} expression is a disjunction of conjunctive
clauses. The elements of the conjuctive clauses are terms. The clauses in the
\ac{DNF} are OR'd together, while the terms in each clauses themselves are
AND'd. The branchsets and each \ac{DNF} expression of the pipeline stage is a
\texttt{JSON array}. \texttt{json-c}\footnote{\url{http://oss.metaparadigm.com/json-c/}} is used to
parse the flow query file, and python scripts have been written to quickly
serialize the python pipeline objects to a \texttt{JSON} query.


The mapping of the \texttt{JSON} query to the structs defined in the execution
engine is not trivial. When reading the \texttt{JSON} query at runtime, the
field offsets of the NetFlow $v5$ record struct are read in as strings.
Utility functions are defined that maps the read names to struct offsets and
read type of each offset and the operations to unique \texttt{enum} members.

The abstract objects that store the JSON query and the results that incubate
from each stage are designed to be self-descriptive and hierarchically
chainable. The complete JSON query information for instance, is held within
the \texttt{flowquery} struct. Each individual branch of the flowquery itself
is described in a \texttt{branch} struct. A collection of these
\texttt{branch} structs are referenced in the parent \texttt{flowquery}
struct.
