The entire JSON flow query is a collection of \ac{DNF} expressions. A \ac{DNF}
expression is a disjunction of conjunctive clauses. The elements of the
conjuctive clauses are terms. The clauses in the \ac{DNF} are OR'd together,
while the terms in each clause themselves are AND'd. The branchsets and each
\ac{DNF} expression of the pipeline stage is a \texttt{JSON array}.
\texttt{json-c} \cite{jsonc} is used to parse the flow query file. The
pipeline stages of the flow query language are also encapsulated in Python
classes and scripts have been written to facilitate the front-end parser to
serialize these pipeline objects to a \texttt{JSON} flow query.  The mapping
of the query to the structs defined in the execution engine is
not trivial. When reading the \texttt{JSON} query at runtime, the field
offsets of the NetFlow $v5$ record struct are read in as strings.  Utility
functions are defined that map the read field names to struct offsets and
field types and the operations to unique enum members.

The abstract objects that store the JSON query and the results that incubate
from each stage are designed to be self-descriptive and hierarchically
chainable.  The complete JSON query information for instance, is held within
the \texttt{flowquery} struct as shown in listing \ref{lst:struct}. Each
individual branch of the flowquery itself is described in a \texttt{branch}
struct.

\lstset{caption=Flow Query and Branch Structs,
				tabsize=2, language=C, numbers=left,stepnumber=1,
        basicstyle=\tiny\ttfamily, numberstyle=\ttfamily\color{gray},
        keywordstyle=\color{blue}, xleftmargin=15pt,
        rulesepcolor=\color{black}, label=lst:struct,
        aboveskip=10pt, captionpos=b}
\begin{lstlisting}
struct flowquery {
  size_t                                      num_branches;
  size_t                                      num_merger_clauses;

  struct branch**                             branchset;
  struct merger_clause**                      merger_clauseset;
  struct merger_result*                       merger_result;
  struct ungrouper_result*                    ungrouper_result;
};

struct branch {
  int                                         branch_id;
  struct ftio*                                ftio_out;
  struct ft_data*                             data;

  size_t                                      num_filter_clauses;
  size_t                                      num_grouper_clauses;
  size_t                                      num_aggr_clause_terms;
  size_t                                      num_groupfilter_clauses;

  struct filter_clause**                      filter_clauseset;
  struct grouper_clause**                     grouper_clauseset;
  struct aggr_term**                          aggr_clause_termset;
  struct groupfilter_clause**                 groupfilter_clauseset;

  struct filter_result*                       filter_result;
  struct grouper_result*                      grouper_result;
  struct groupfilter_result*                  gfilter_result;
};
\end{lstlisting}

The \texttt{JSON} query can also trigger and disable the stages at runtime.
This means that one only has to supply the constructs that one wishes to use.
The constructs that are not defined in the \texttt{JSON} query are inferred by
the execution engine as a disable request. The execution engine uses disable
flags that are turned on when the query is parsed. These flags are used
throughout the engine to only enable the requested functionality.

