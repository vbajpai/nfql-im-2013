Each stage of the processing pipeline is expressed in the JSON query as a
\ac{DNF} expression. The elements of the conjuctive clauses are terms.
\texttt{json-c} \cite{jsonc} is used to parse the flow query file. The
pipeline stages of the flow query language are also encapsulated in Python
classes and scripts have been written to facilitate the front-end parser to
serialize these pipeline objects to a JSON flow query.  The mapping
of the query to the structs defined in the execution engine is not trivial.
When reading the JSON query at runtime, the field names of a NetFlow
$v5$ record are read in as strings.  Utility functions are defined that map
the field names to internal struct offsets and the field types and the
operations to internal unique enum members.

The abstract objects that store the JSON query and the results that incubate
from each stage are designed to be self-descriptive and hierarchically
chainable.  The complete JSON query information for instance, is held within
the \texttt{flowquery} struct as shown in listing \ref{lst:struct}. Each
individual branch of the flowquery itself is described in a \texttt{branch}
struct.

\lstset{caption=Flow Query and Branch Structs,
				tabsize=2, language=C, numbers=left,stepnumber=1,
        basicstyle=\tiny\ttfamily, numberstyle=\ttfamily\color{gray},
        keywordstyle=\color{blue}, xleftmargin=15pt,
        rulesepcolor=\color{black}, label=lst:struct,
        aboveskip=10pt, captionpos=b}
\begin{lstlisting}
struct flowquery {
  size_t                                      num_branches;
  size_t                                      num_merger_clauses;

  struct branch**                             branchset;
  struct merger_clause**                      merger_clauseset;
  struct merger_result*                       merger_result;
  struct ungrouper_result*                    ungrouper_result;
};

struct branch {
  int                                         branch_id;
  struct ftio*                                ftio_out;
  struct ft_data*                             data;

  size_t                                      num_filter_clauses;
  size_t                                      num_grouper_clauses;
  size_t                                      num_aggr_clause_terms;
  size_t                                      num_groupfilter_clauses;

  struct filter_clause**                      filter_clauseset;
  struct grouper_clause**                     grouper_clauseset;
  struct aggr_term**                          aggr_clause_termset;
  struct groupfilter_clause**                 groupfilter_clauseset;

  struct filter_result*                       filter_result;
  struct grouper_result*                      grouper_result;
  struct groupfilter_result*                  gfilter_result;
};
\end{lstlisting}

The JSON query can also disable the stages at runtime.  This means
that one only has to supply the constructs that one wishes to use.  The
constructs that are not defined in the JSON query are inferred by the
execution engine as a disable request. The execution engine uses disable flags
that are turned on when the query is parsed. These flags are used throughout
the engine to only enable the requested functionality.

