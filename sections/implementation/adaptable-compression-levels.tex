The flow-records echoed to the standard output can also be written to a
Netflow $v5$ \texttt{flow-tools} file. The \texttt{--dirpath} switch allows
one to provide a directory path where the results can be stored.  Each stream
is stored as its own file with an ID to disambiguate it. Results from each
stage of the pipeline can also be written to separate files with the increase
in the verbosity level. In fact, \texttt{--dirpath} and \texttt{--verbose}
work well together to adjust the level at which the writes are to be made.

\begin{figure}[h!]
  \begin{center}
    \includegraphics* [width=1.0\linewidth]{zlevel}
    \caption{F$(v2)$: \texttt{z-level} Effect on Performance}
    \label{fig:engine-zlevel}
  \end{center}
\end{figure}

The engine uses the \texttt{zlib} \cite{rfc1950} software library to compress
the results written to the \texttt{flow-tools} files. \texttt{zlib} supports
$9$ compression levels with $9$ being the highest. The \ac{NFQL} engine
supports \texttt{--zlevel} switch to allow the user to supply its desired
choice of the of the compression level. A default level of $5$ is used for
writes if the switch is not supplied during runtime.  Fig.
\ref{fig:engine-zlevel} shows the time taken to write a sample of records
passing the filter stage for each \texttt{z-level}. It goes to show that each
level adds its own performance overhead and must be used with discretion.
