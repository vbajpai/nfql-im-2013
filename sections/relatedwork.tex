%The study of the network traffic patterns began with tools that utilized raw
%packet captures at the monitoring point for traffic measurement.
%\texttt{tcpdump} and \texttt{wireshark} have since then become the de-facto
%tools for packet capture and analysis. \texttt{tcpdump} \cite{tcpdump-manpage}
%is a premier command-line utility that uses the \texttt{libpcap}
%\cite{pcap-manpage} library for packet capture. The power of \texttt{tcpdump}
%comes from the richness of its expressions, the ability to combine them using
%logical connectives and extract specific portions of a packet using filters.
%\texttt{wireshark} \cite{wireshark-manpage} is a GUI application, aimed at
%both journeymen and packet analysis experts. It supports a large number of
%protocols, has a straightforward layout, excellent documentation, and can run
%on all major operating systems.

In recent years, a number of tools have been developed that can capture the
traffic as flow records and use them for network analysis.

Simple analysis of network traffic can be done by a range of graphical
utilities like ntop \cite{ntop:2000}, FlowScan \cite{flowscan:2000}, NfSen
\cite{phaag:2006} and Stager \cite{oslebo:2006}. All these tools understand the
NetFlow format while ntop and Stager can also process \ac{IPFIX}. They use a
round robin database to store flow information, except for Stager which uses a
MySQL database. Graphical presentation is done via a webinterface or in the
case of FlowScan via RRDTool.

\texttt{flow-tools} and \texttt{nfdump} are among the most popular tools used
for analyzing NetFlow data. \texttt{flow-tools} \cite{sromig:2000} is a suite
of programs for capturing and processing NetFlow v5 flow records. It consists
of 24 separate tools that work together by connecting them via UNIX pipes. It
can capture, read, filter, and print flow records internally saved in a
fixed-size format.  \texttt{nfdump} \cite{phaag:2006} is a very similar tool
that uses a different storage format. Flow records are captured using
\texttt{nfcapd} and then processed by \texttt{nfdump}. The power of filtering
rules in both the tools is however mostly limited to absolute comparisons of
flow attributes. As a result, relative comparison amongst different flows or
querying a timing relationship among them is not possible.  SiLK \cite{SiLK} is
a network traffic collection and analysis tool developed and maintained by the
CERT Network Situational Awareness Team (CERT NetSA) at Carnegie Mellon
University.  SiLK is the tool that comes quite close to providing similar
capabilities as provided by \ac{NFQL} and is therefore used as a reference
point to compare the performance of the \ac{NFQL} execution engine in this
paper. The design and implementation of SiLK, however, differs a lot from that
of \ac{NFQL}. For instance, in SiLK there are separate tools to perform the
task of each stage of the \ac{NFQL} processing pipeline. The stage
functionality is not full-fledged though. The grouping and merging operations
can only be performed using an equality operator. This restriction allows the
tool to perform optimization such as using hash tables to perform lookups.
There are also stringent requirements to how flow-data needs to be organized
before it can be piped into a tool. The grouping tool, for instance, assumes
that the to-be supplied input flow data is already sorted on the field column.
These requirements can make it a little cumbersome to design a full-fledged
flow query.  For instance, trying to mimic a \ac{NFQL} query in SiLK sometimes
ends up as a shell script with over a dozen of SiLK tools piped together.
