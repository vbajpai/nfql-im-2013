\texttt{flow-tools} \cite{sromig:2000} is a suite of programs for capturing
and processing NetFlow v5 flow records. It consists of 24 separate tools that
work together by connecting them via UNIX pipes. \todo{...}

\texttt{nfdump} \cite{phaag:2006} works similar to flow-tools but uses a
different storage format. Flow records are captured using \texttt{nfcapd} and
then processed by \texttt{nfdump} which can filter as well as display the
sorted and filtered result. The power of its filtering rules is similar to
that of flow-tools and as such is mostly limited to absolute comparisons of
flow attributes.

\texttt{tcpdump} and \texttt{wireshark} are the most popular tools used for
packet capture and analysis. \texttt{tcpdump} \cite{tcpdump-manpage} is a
premier command-line utility that uses the \texttt{libpcap}
\cite{pcap-manpage} library for packet capture. The power of \texttt{tcpdump}
comes from the richness of its expressions, the ability to combine them using
logical connectives and extract specific portions of a packet using filters.
\texttt{wireshark} \cite{wireshark-manpage} is a GUI application, aimed at
both journeymen and packet analysis experts. It supports a large number of
protocols, has a straightforward layout, excellent documentation, and can run
on all major operating systems.

