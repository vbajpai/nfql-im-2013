In recent years, a number of tools have been developed that can capture the
traffic as flow records and use them for network analysis. Simple analysis of
network traffic can be done by a range of graphical utilities like ntop
\cite{ntop:2000}, FlowScan \cite{flowscan:2000}, NfSen \cite{phaag:2006} and
Stager \cite{oslebo:2006}. All these tools understand the NetFlow format while
ntop and Stager can also process \ac{IPFIX}. \texttt{flow-tools} and
\texttt{nfdump} are among the most popular tools used for analyzing NetFlow
data. \texttt{flow-tools} \cite{sromig:2000} is a suite of programs for
capturing and processing NetFlow v5 flow records. It consists of 24 separate
tools that work together by connecting them via UNIX pipes. It can capture,
read, filter, and print flow records internally saved in a fixed-size format.
\texttt{nfdump} \cite{phaag:2006} is a very similar tool that uses a different
storage format. The power of filtering rules in both the tools is however
mostly limited to absolute comparisons of flow attributes. As a result,
relative comparison amongst different flows or querying a timing relationship
among them is not possible.


SiLK \cite{SiLK} is a network traffic collection and analysis tool that comes
quite close to providing similar capabilities as \ac{NFQL} and is therefore
used as a reference point to compare the performance of the \texttt{nfql}
execution engine. The design and implementation of SiLK, however, differs a
lot from that of \texttt{nfql}. For instance, in SiLK there are separate tools to
perform the task of each stage of the \ac{NFQL} processing pipeline. The stage
functionality is not full-fledged though. The grouping and merging operations
can only be performed using an equality operator. This restriction allows the
tool to perform optimization such as using hash tables to perform lookups.
\ac{NFQL} on the other hand, provides a much richer set of comparisons
operations, such as \emph{equal, not equal, greater than, less than, greater
or equal, less or equal}. There are also stringent requirements in SiLK how
flow-data needs to be organized before it can be piped into a tool. The
grouping tool, for instance, assumes that the to-be supplied input flow data
is already sorted on the field column.  These requirements can make it a
little cumbersome to design a full-fledged flow query.  For instance, trying
to mimic a \ac{NFQL} query in SiLK sometimes ends up as a shell script with
over a dozen of SiLK tools piped together.
