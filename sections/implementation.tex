The \texttt{nfql} architecture primarily consists of a front-end parser backed
up by an execution engine as shown in Fig. \ref{fig:nfql-pipeline}. The
execution engine is the brain of \texttt{nfql} where the complete pipeline is
processed. It receives the flow query at runtime using a JSON \cite{rfc4627}
intermediate format. The execution engine is written in C, however the
front-end parser can be written in any desired language.

\subsection{Flow Query Intermediate Format}
A \ac{DNF} is a disjunction of conjunctive clauses. The elements of the
conjuctive clauses are terms. Each stage of the pipeline is represented in the
flowquery as a \ac{DNF} expression. The clauses in the \ac{DNF} are OR'd
together, while the terms in each clauses themselves are AND'd. The abstract
objects that store the JSON query and the results that incubate from each
stage are designed to be self-descriptive and hierarchically chainable. The
complete JSON query information for instance, is held within the
\texttt{flowquery} struct. Each individual branch of the flowquery itself is
described in a \texttt{branch} struct. A collection of these \texttt{branch}
structs are referenced in the parent \texttt{flowquery} struct.


\label{subsec:intermediate-format}

\subsection{Execution Workflow}
\subsubsection{Splitter} A custom C library was written to directly read/write
data in the \texttt{flow-tools} format. The library sequentially reads the
complete flow-records into memory to support random access required for
relative filtering. Each flow-record is stored in a \texttt{char} array and
the offsets to each field are stored in separate \texttt{structs}.  The array
of such records are indexed allowing fast retrieval in $O(1)$ time. This
allowed the flow-records to be easily referenced by an identifier, thereby
giving away the need to every time copy all the flow-records when moving ahead
in the processing pipeline.

In the default method, there is a comparison function defined for every
possible field length $(33)$ and comparison operations $(19)$. These functions
are generated using a Python script. The rule definitions are now able to make
calls using a function name derived from the combination of field length,
delta type and operation.  This subverts the need to define complex branching
statements and reduces complexity.

\subsubsection{Filter}
\begin{figure}[h!]
  \begin{center}
    \includegraphics* [width=1.0\linewidth]{filter-fv1-fv2}
    \caption{Filter Stage: F$(v1)$ vs F$(v2)$}
    \label{fig:fv1-fv2-filter}
  \end{center}
\end{figure}

The flow query language suggests to read all the flow records of a supplied
trace into memory before starting the processing pipeline.  Since, the filter
stage uses the supplied set of absolute rules to make a decision on whether or
not to keep a flow record; it has to pass through the whole in-memory
recordset \emph{again} to fill in the filter results. This technique involves
multiple linear runs on the trace and therefore slows down when the ratio of
number of filtered records to the total number of flow-records is high. The
\ac{NFQL} implementation merges the filter stage with in-memory read of the
trace. This means, a decision on whether or not to make room for a record in
memory and eventually hold a pointer for it in filter results is done upfront
as soon as the record is read from the trace. In addition, if a request to
write the filter stage results to a flow-tools file has been made, the writes
are also made as soon the filter stage decision is available, thereby allowing
reading-filtering-writing to happen in $O(n)$ time, where $n$ is the number of
records in the trace. It is important to note that the filtered records are
saved in common location from where they are referenced by each branch. This
helps keep the memory costs at a minimum when multiple branches are involved.
A publically available Netflow $v5$ trace
\footnote{\url{http://traces.simpleweb.org/traces/netflow/netflow1/netflow000.tar.bz2}}
was used to compare the performance of the new filter in F$(v2)$ with that
from F$(v1)$ as shown in Fig. \ref{fig:fv1-fv2-filter}. The trace has around
$20M$ flow-records.

\subsubsection{Grouper}

In order to be able to make comparisons on field offsets, a simple approach is
to linearly walk through each filtered record against the filtered recordset
leading to a complexity of $O(n^2)$, where $n$ is the number of filtered
records. A smarter approach is to put the copy in a hash table and then try to
map each pointer while walking down the filtered recordset once, leading to a
complexity of $O(n)$. The hash table approach, although will work on this
specific example, will fail badly on other relative comparisons. A better
approach, as implemented in F$(v2)$ is to sort the filtered recordset on all
the requested grouping rules. This helps the execution engine perform a
nested \texttt{bsearch} to reduce the linear pass to a fairly small filtered
recordset. However, the number of elements upon which the search has to be
performed needs to be known at each level of the binary lookup. In order to
avoid a linear run to count the number of elements, the parent level
\texttt{bsearch} invocation returns a boundary with the first and last element
to enable such a calculation. This helps the grouper perform faster search
lookups to find records that must group together in $O(n*lg(k))$ time with a
preprocessing step taking $O(n*lg(n)) + O(n)$ in the average case, where $n$
is the number of filtered records and $k$ is the number of unique filtered
records.

\begin{figure}[h!]
  \begin{center}
    \includegraphics*[width=1.0\linewidth]{grouper-fv2-bsearch-fv2}
    \caption{Grouper Stage: F$(v2)$ (Generic) vs F$(v2)$ (Specific)}
    \label{fig:fv1-fv2-grouper}
  \end{center}
\end{figure}

The grouping approach has further been optimized when the filtered records are
grouped for equality. In such a scenario, the need to search for unique
records and a  subsequent binary search goes away.  The groups can now be
formed in $O(n)$ time with a more involved preprocessing step taking
$O(p*n*lg(n)$ where is $n$ is the number of filtered records, and $p$ is the
number of grouping rules. The performance evaluation of the grouper handling
this special case against its behavior when handling generic cases is shown in
Fig. \ref{fig:fv1-fv2-grouper}. Profiling the performance of the grouper on
queries that produce higher ratios reveal that a significant amount of time is
taken in \texttt{bsearch}. The special case of equality comparisons eliminate
these calls and further reduce the time in the long tail as shown in Fig.
\ref{fig:fv1-fv2-grouper}.

\subsubsection{Group Aggregations} Group records are a conglomeration of
several flow records with some common characteristics defined by the flow
query. Some of the non-common characteristics can also be aggregated into a
single value using group aggregations. It is useful if a single group record
can be again mapped into a NetFlow $v5$ record template, so that it can be
written to file as a representative of all its members.

\subsubsection{Group Filter} The groupfilter is used to filter the groups
produced by the grouper based on some absolute rules defined in a \ac{DNF}
expression. The \texttt{struct} term holds information about the flow record
offset, the value being compared to and the operator which maps to a unique
\texttt{enum} value.  This enum value is used to map the operation to a
specific group-filter function. The group-filter functions are auto-generated
using a python script.




\subsubsection{Merger}

The merger is used to relate groups from different branches according to a
merging criterean. The implementation is not trivial since the number of
branches that need to be spawned is read from the query and is not known until
compile time. As a result, an iterator that can provide all possible
permutations of $m-$tuple (where $m$ is the number of branches) group record
$ID$s was needed.  The result of the iterator can then be used to make a
match. The merger stage, begins by initializing this iterator passing it the
number of branches, and information about each branch. Then, it loops over to
get a new $m-$tuple of group record $ID$s on each iteration until the iterator
returns \texttt{false}.

The merger as formulated in the flow query language needs to match each group
record from one branch with every other record of each branch. This leads to a
complexity of $O(n^m)$ where $n$ is the number of filtered group records and
$m$ is the number of branches. The possible number of tries when matching
group records however can be reduced by sorting the group records on the field
offsets used for a match. The \ac{NFQL} implementation optimizes the merger to
skip over iterator permutations when a state of a current field offset value
may not allow any further match beyond the index in the current branch.  For
such an optimization to work, the filtered group records must be sorted in the
order of field offsets specified in the merger clause. Specifying the filtered
group records in any other order may lead to undefined behavior. This means,
that if the same field offsets were used in the grouper stage, the terms in
the group clause can be rearranged by the query designer to align with the
order of terms in the merger clause.

The flow query language also bases the merger matches on the notion of matched
tuples. This means that a filtered group record can be written to a file
multiple times if it is part of multiple matched tuples. This situation is
very common and it worsens when different branches have similar filtered
groups records. Since, the function of the merger is to find a match of groups
records across branches based on a predefined condition, all the group records
across branches that satisfy the condition can be clubbed into one collection
instead of separate tuples. All the group records within a collection can then
be written to the file. This eliminates the inherent redundancy and
significantly improves the merger performance.

The performance comparison of this approach against the one suggested by the
specification is tricky. The merger implementation of the original
specification is slow. It is so slow that it keeps churning the CPU for days
without results. The newer approach takes less than an hour. The performance
evaluation of this newer approach is discussed in more details in the next
section

\subsubsection{Ungrouper} The approach of clubbing the merged group records into
a collection incurs a reimplementation of the ungrouper. The ungrouper, as a
result accepts a collection of matched filtered group records as input. It
then iterates over each collection to unfold it groups and write their flow
record members to files.


\label{subsec:execution-workflow}

\subsection{Performance Optimizations}
There can be a situation where the query designer may incorrectly ask for
aggregation on a field already specified in a grouper (or filter) clause. If
the relative operator is an equality comparison, the aggregation on such a
field becomes less useful, since the members of the grouped record will always
have the same value for that field. \ac{NFQL} realizes this redundant request
and ignores such aggregations.

\ac{NFQL} has dedicated comparator functions for each type of operation and
the type of the field offset it operates upon. It is not guaranteed that given
the type of the query and trace, the program will eventually go through each
stage of the pipeline. It is also possible that the program exits before,
because there is nothing more for the next stage to compute. The function
pointers are therefore set as late as possible and are called from their
respective stages just before the comparison is needed.

Each stage of the processing pipeline is dependent on the result of the
previous one. As a result, the stages should only proceed, when the previous
returned results. Implementing such a response was straightforward for the
grouper and group filter, the merger although was a little trickier.  The
merger stage proceeds only when every branch has non-zero filtered groups.
The iterator initializer deallocates and returns \texttt{NULL} if any one
branch has $0$ filtered groups.  Consequently a check is performed in the
merger to make sure \texttt{iter} is \emph{not} \texttt{NULL}.

The flow-records echoed to the standard output can also be written as
\texttt{flow-tools} files. In fact, results from each stage of the pipeline
can be written to separate files with the increase in the verbosity level.
This leads to additional loops over the resultsets if the writes are made at
the end of the processing pipeline. The execution engine therefore writes each
result record to a file as soon as it is seen by the pipeline stage.

\label{subsec:performance-optimizations}

\subsection{Adaptable Compression Levels}
\begin{figure}[h!]
  \begin{center}
    \includegraphics* [width=0.9\linewidth]{zlevel}
    \caption{\texttt{z-level} Effect on Performance}
    \label{fig:engine-zlevel}
  \end{center}
\end{figure}

The engine uses the \texttt{zlib} \cite{rfc1950} software library to compress
the results written to the \texttt{flow-tools} files. \texttt{zlib} supports
$9$ compression levels with $9$ being the highest. \texttt{nfql} allows the
user to supply its desired choice of the compression level.  A default level
of $5$ is used for writes if a choice is not indicated.  Fig.
\ref{fig:engine-zlevel} shows the time taken to write a sample of records
passing the filter stage for each \texttt{z-level}. It goes to show that each
level adds its own performance overhead and must be used with discretion. It
is also important to note that other tools may use different compression
algorithms.  \texttt{nfdump} for instance, uses \texttt{lzo} \cite{lzo}
compression to trade space for faster compression and decompression.

\label{subsec:adaptable-compression-levels}
