\documentclass[10pt, conference]{IEEEtran}

% IEEEtran package options
%----------------------------------------------------------------------
% *** CITATION PACKAGES ***
\usepackage{cite}

% *** GRAPHICS RELATED PACKAGES ***
\usepackage[pdftex]{graphicx}
\graphicspath{{figures/}}
\DeclareGraphicsExtensions{.pdf,.eps}
%----------------------------------------------------------------------



% added by Vaibhav
%----------------------------------------------------------------------
\usepackage[utf8]{inputenc}
\usepackage[T1]{fontenc}
\usepackage[nolist]{acronym}
\usepackage{url}
\usepackage{listings}
\usepackage{todonotes}
\renewcommand{\ttdefault}{cmtt}
%----------------------------------------------------------------------

\begin{document}

\title{NFQL: An Efficient Tool for Querying Network Flow Records}
\author{\IEEEauthorblockN{Vaibhav Bajpai, Johannes Schauer, Jürgen
Schönwälder} \IEEEauthorblockA{Computer Science, Jacobs University Bremen,
Germany \\ \texttt{\{v.bajpai, j.schauer,
j.schoenwaelder\}@jacobs-university.de}}}

\maketitle

\begin{acronym}
  \acro{NFQL}{Network Flow Query Language}
  \acro{DNF}{Disjunctive Normal Form}
  \acro{IPFIX}{Internet Protocol Flow Information Export}
  \acro{MPLS}{Multiprotocol Label Switching}
\end{acronym}


\begin{abstract} Cisco's NetFlow protocol and IETF's IPFIX open standard are
  the de-facto techniques for collecting aggregate network traffic statistics.
  However, understanding certain patterns in these network statistics requires
  sophisticated flow analysis tools that can mine flow records for such a
  usage. We recently proposed a flow query language (NFQL) that can process
  flow records, aggregate them into groups, apply absolute or relative
  filters, invoke Allen interval algebra rules, and merge group records. In
  this paper, we introduce an efficient implementation of the query language.
  The implementation has been evaluated by a suite of benchmarks against
  contemporary flow-processing tools.\end{abstract}
% no keywords



% sections
%----------------------------------------------------------------------
\section{Introduction}
Researchers, service providers and security analysts have long been interested
in network and user behavioral patterns of the traffic crossing the internet
backbone. They want to use this information for the purpose of billing and
mediation, bandwidth provisioning, detecting malicious attacks, network
performance evaluation and overall improvement. Traffic measurement techniques
that have been rapidly evolving in the last decade, have matured enough today
to provide such an insight.

Flow capture today, has emerged out to be one of the favored network
measurement techniques. In this technique, packets traversing a monitoring
point are not captured raw, instead they are aggregated together based on some
common characteristics. The common characteristics are learnt by inspecting
the packet headers as they cross the monitoring point. Flow-records resulting
from such an aggregation are then exported to a collector for further
analysis. This not only reduces the amount of traffic at the monitoring point,
but also provides fine-grained control over the network data which was not
previously possible using SNMP interface-level queries.

NetFlow and \ac{IPFIX} are the two popular standards of IP flow information
export. NetFlow \cite{rfc3954} is a proprietary network protocol designed by
Cisco Systems. It allow routers to generate and export flow records to a
designated collector. NetFlow v$9$ provides flexibility of user-tailored
export templates, \ac{MPLS} and IPv$6$ support and a larger set of flow keys.
\ac{IPFIX} \cite{rfc5101} on the other hand is an open standard by IETF deemed
to be the logical successor of NetFlow v$9$ on which it is based. The novelty
of the standard lies in its ability to describe record formats at runtime
using templates based on an extensible and well-defined information model. The
data transfer mechanism is also simplistic and extensible by being
unidirectional and transport protocol agnostic.

The wide applicability of this approach is easily seen from the pervasive use
of flow records for a vibrant set of network analysis applications. For
instance, the authors in \cite{mkim:2004} use the flow characteristics in the
traffic pattern to formalize a detection function that maps traffic patterns
to different DoS attacks, whereas in \cite{sdominik:2010}, the authors
use the flow-record data to exploit timing characteristics of webmail clients
to classify features that could identify webmail traffic from any other
traffic running over HTTPS.

It goes to show that understanding these intricate traffic patterns require
sophisticated flow analysis tools that can mine flow records for such a usage.
Unfortunately current tools fail to deliver owing to their poor language
design and simplistic filtering methods.  We recently proposed a flow query
language design \cite{vmarinov:thesis:2009} that aims to cater to such needs.
It can process flow records, aggregate them into groups, apply absolute (or
relative) filters and invoke Allen interval algebra rules \cite{fallen:1983}.

In this paper, we introduce \ac{NFQL}, an efficient C implementation of the
flow query language. It is the next iteration of the first prototype
implementation, Flowy \cite{kkanev:2010}, which was written in Python.
\ac{NFQL} however is not just a reimplementation in a new language, but
the inner workings have been reimagined to allow the execution engine to
scale to real-world sized traces. This has been possible by replacing the
performance hit stages of the pipeline with crispier algorithms that lower
down the execution times to factor making them comparable to contemporary
flow analysis tools.

The rest of the paper is organized as follows. In section \ref{sec:design} we
describe the flow query language by discussing each stage of the processing
pipeline. In section \ref{sec:implementation} we introduce the inner workings
of \ac{NFQL}. It begins by providing an overview of the \ac{NFQL} components
and the structure of the intermediate format used to exchange messages. The
workflow of the execution engine is described next and is supplemented by a
number of performance optimizations made to make the implementation scale. It
is followed by performance evaluation results by comparing them against
contemporary flow processing tools in section \ref{sec:evaluation} and
concludes with a discussion on its current limitation and future outlook in
section \ref{sec:conclusion}.

\begin{figure*}[!t]
\centering
\includegraphics[width=1.0\linewidth]{nfql-pipeline}
\caption{NFQL Processing Pipeline \cite{vmarinov:2009}}
\label{fig:nfql-pipeline}
\end{figure*}
\label{sec:introduction}
\section{Related Work}
\texttt{flow-tools} \cite{sromig:2000} is a suite of programs for capturing
and processing NetFlow v5 flow records. It consists of 24 separate tools that
work together by connecting them via UNIX pipes. \todo{...}

\texttt{nfdump} \cite{phaag:2006} works similar to flow-tools but uses a
different storage format. Flow records are captured using \texttt{nfcapd} and
then processed by \texttt{nfdump} which can filter as well as display the
sorted and filtered result. The power of its filtering rules is similar to
that of flow-tools and as such is mostly limited to absolute comparisons of
flow attributes.

\texttt{tcpdump} and \texttt{wireshark} are the most popular tools used for
packet capture and analysis. \texttt{tcpdump} \cite{tcpdump-manpage} is a
premier command-line utility that uses the \texttt{libpcap}
\cite{pcap-manpage} library for packet capture. The power of \texttt{tcpdump}
comes from the richness of its expressions, the ability to combine them using
logical connectives and extract specific portions of a packet using filters.
\texttt{wireshark} \cite{wireshark-manpage} is a GUI application, aimed at
both journeymen and packet analysis experts. It supports a large number of
protocols, has a straightforward layout, excellent documentation, and can run
on all major operating systems.

\label{sec:relatedwork}
\section{Flow Query Language}
The pipeline consists of a number of independent processing elements that are
connected to one another using UNIX-based pipes. Each element receives the
content from the previous pipe, performs an operation and pushes it to the
next element in the pipeline. Fig. \ref{fig:nfql-pipeline} shows an
overview of the processing pipeline. A complete description on the semantics
of each element in the pipeline can be found in \cite{vmarinov:2009}

The splitter takes the flow-records data as input in the \texttt{flow-tools}
compatible format. It is responsible to duplicate the input data out to
several branches without any processing whatsoever. This
allows each of the branches to have an identical copy of the flow data to
process it independently.

The filter performs \emph{absolute} filtering on the input flow-records data.
The flow-records that pass the filtering criterion are forwarded to the
grouper, the rest of the flow-records are dropped. The
filter compares separate fields of a flow-record against either a constant
value or a value on a different field of the \emph{same} flow-record. The
filter cannot \emph{relatively} compare two different incoming flow-records

The grouper performs aggregation of the input flow-records data. It consists
of a number of rule modules that correspond to a specific subgroup. A
flow-record in order to be a part of the group should be a part of at-least
one subgroup. A flow-record can be a part of multiple subgroups within a
group. A flow-record cannot be part of multiple groups.
The grouping rules can be either absolute or relative. The newly formed groups
which are passed on to the group filter can also contain meta-information
about the flow-records contained within the group using the aggregate clause
defined as part of the grouper query.

The group-filter performs \emph{absolute} filtering on the input group-records
data. The group-records that pass the filtering criterion are forwarded to the
merger, the rest of the group-records are dropped. The group-filter compares
separate fields (or aggregated fields) of a
flow-record against either a constant value or a value on a different field of
the \emph{same} flow-record. The group-filter cannot \emph{relatively} compare
two different incoming group-records

The merger performs relative filtering on the N-tuples of groups formed from
the N stream of groups passed on from the group-filter as input. The merger
rule module consists of a number of a submodules, where the output of the
merger is the set difference of the output of the first
submodule with the union of the output of the rest of the submodules. The
relative filtering on the groups are applied to express timing and concurrency
constraints using Allen interval algebra \cite{fallen:1983}

The ungrouper unwraps the tuples of group-records into individual
flow-records, ordered by their timestamps. The duplicate
flow-records appearing from several group-records are eliminated and are sent
as output only once.

\label{sec:design}
\section{Implementation}\label{sec:implementation}
\subsection{Splitter}
\subsection{Filter}
\subsection{Grouper}
\subsubsection{Group Aggregations}
\subsection{Group Filter}
\subsection{Merger}
\subsection{Ungrouper}

\section{Performance Evaluation}
We used the first $20M$ records from the publically available Trace 7 from the
SimpleWeb \cite{simpleweb} repository for our performance evaluations. The
input trace was compressed at \texttt{ZLIB\_LEVEL} $5$ using the \texttt{zlib}
suite. It was also converted to \texttt{nfdump} and SiLK compatible formats
keeping the same compression level. The suite was run on a machine with $24$
cores of $2.5$ GHz clock speed and $18$ GiB of memory.

The first set of queries stress the filter stage.  We use varying values on
the \texttt{packet} field offset to control the amount of flow records that
are passed by the filter. The resultant filtered records are written as
\texttt{flow-tools} files and compressed at \texttt{ZLIB\_LEVEL} $5$. The
ratio of the number of filtered records in the output trace to the number of
the flow records in the input trace is plotted against time. The results are
shown in Fig. \ref{fig:benchmarks-filter}.

\begin{figure}[h!]
  \begin{center}
    \includegraphics* [width=0.9\linewidth]{filter}
    \caption{Filter Stage: NFQL vs SiLK, Flow-Tools, Nfdump}
    \label{fig:benchmarks-filter}
  \end{center}
\end{figure}

It can be seen that the performance of the filter stage in \texttt{nfql} is
comparable to that of \texttt{flowtools} and SiLK. SiLK takes less time on
lower ratios, but then SilK and \texttt{nfdump} also use their own file
format. \texttt{nfdump} appears to be significantly faster than the rest. This
is because \texttt{nfdump} uses the \texttt{lzo} compression scheme. Adding
\texttt{lzo} compression will likely improve \texttt{nfql}'s filter
performance. Note that all the tools were single-threaded
in this evaluation, and did not completely utilize the $24$ available cores. 

The second set of queries stress the grouper stage. We reuse the filter query
that produces a $1.0$ ratio to allow the grouper to receive the entire trace
as a filtered recordset. The grouper part of the query then gradually
increases the number of grouping terms in the \ac{DNF} expression to increase
the output/input ratio. The resultant groups are again written as
\texttt{flowtools} files using the same \texttt{zlib} compression level. The
ratio of the number of groups formed to the number of the input filtered
records is plotted against time. \texttt{nfdump} and \texttt{flowtools} do not
support grouping, and therefore are not considered in this evaluation. The
results are shown in Fig. \ref{fig:benchmarks-grouper}.

\begin{figure}[ht!]
  \begin{center}
    \includegraphics* [width=0.9\linewidth]{grouper}
    \caption{Grouper Stage: NFQL vs SiLK}
    \label{fig:benchmarks-grouper}
  \end{center}
\end{figure}

The evaluation graph reveals that the time taken by the tools are comparable
on lower ratios, but on higher ratios, \texttt{nfql} starts to drift apart.
Since most of the time is taken in writing the records to files, it is unclear
whether SiLK's usage of its own file format is responsible for the drift.
SiLK's \texttt{rwgroup} tool is also supplied a \texttt{-{}-summarize} flag to
force it to write only the first record of each group, to make both tools
write the same number of records. This gives SiLK the leverage to not store
information about which members are part of the group. \texttt{nfql} on the
other hand needs to allocate resources (which may take time) to keep this
information in its data structures, since the ungrouper later may request to
write the members of a group while unfolding the tuples.  It is also important
to note that both the tools again remained single-threaded throughout the
evaluation. SiLK took advantage of an inherent concurency arising from a pipe
between \texttt{rwsort} and \texttt{rwgroup}, which makes the two processes
run concurrently, the effect of which gets more pronounced on higher ratios.
The profiling results from GNU \texttt{gprof} \cite{graham:1982} indicate that
$60\%$ of the time is taken in \texttt{qsort} comparator calls.  As a result,
it comes as no surprise, that bifurcating \texttt{qsort} invocation to
multiple threads and later merging the results back using merge sort will help
parallelize the grouper stage and maybe reduce the drift on higher ratios. In
addition, since all of the evaluation queries had grouping terms using an
equality comparator, \texttt{nfql} can introspect such a grouping rule to
dynamically optimize processing searches using a hashtable and turn to
\texttt{qsort} based grouping only as a fallback.

The third set of queries stress the group filter stage. We reuse
the filter and grouper queries that produce a $1.0$ ratio to allow the group
filter to receive the entire trace as input. This means, each flow record of
the original trace now becomes a group record for the group filter.  The group
filter then reuses the same varying values of the \texttt{packet} field offset
to control the amount of groups that are filtered ahead. The filtered groups
are again written as \texttt{flowtools} files using the same \texttt{zlib}
compression level. The ratio of the number of output filtered groups to the
number of the input group is plotted against time. The results are shown in
Fig. \ref{fig:benchmarks-groupfilter}.

\begin{figure}[ht!]
  \begin{center}
    \includegraphics* [width=0.9\linewidth]{groupfilter}
    \caption{Group Filter Stage: NFQL vs SiLK}
    \label{fig:benchmarks-groupfilter}
  \end{center}
\end{figure}

It can be seen that the timings of \texttt{nfql} are far apart from that of
SiLK.  It is due to the drift already created by the grouper at the $1.0$
ratio in the previous stage. As a result, the group filter comes into play
only after $300$ seconds, whereas SiLK's group filtering already starts just
below $150$ seconds. Even if we normalize the graph, it can be observed that
the \texttt{nfql} group filter has a higher slope. This is because it is only
executed once the grouper returns, and therefore has to reiterate the groups
to make a filtering decision.

The fourth set of queries stress the merger stage. We reuse the filter,
grouper and group filter queries that produce a $1.0$ ratio. These queries are
then run in two separate branches to produce identical filtered group records.
The merger then applies match rules to produce different output to input
ratios. The groups that are merged are again written as \texttt{flowtools}
files using the same \texttt{zlib} compression level. The ratio of the number
of merged groups to twice\footnote{Each branch pushes the entire trace as an
input to the merger.} the number of flow records in the original trace is
plotted against time. A data point for SiLK for the $0.2$ ratio is not
available since the \ac{NFQL} query executed at that data point uses
\texttt{OR} expressions, which are not supported by SiLK.

\begin{figure}[ht!]
  \begin{center}
    \includegraphics* [width=0.9\linewidth]{merger}
    \caption{Merger Stage: NFQL vs SiLK}
    \label{fig:benchmarks-merger}
  \end{center}
\end{figure}

It can be seen that the merger is the most performance critical stage of the
\ac{NFQL} pipeline thus far. It is due to the fact that the merger is working
on twice the number of flow records than any other previous stage. In
addition, each branch is writing the results of the filter, grouper and group
filter stage to \texttt{flowtools} files. As a result, the amount of disk I/O
involved is twice as much as well. Even though each branch is delegated to a
separate core using affinity masks, most of time is taken in writing these
results to the file. These results although look less promising, they are way
better than the original \texttt{nfql} merger implementation. The optimized
merger implementation takes advantage of sorted nature of filtered groups and
therefore can significantly reduce the number of merger matches. It also
writes a merged group record to a file only once despite the number of times
it has matched. Without these optimizations, running such queries on the
merger kept the CPU churning for days without results.

The last set of queries stress the ungrouper stage. They reuse the entire
merger queries as is, but enable the ungrouper as well. This means, that the
ungrouper now attempts to unfold the merged groups returned by the merger to
write their member flow records to \texttt{flowtools} files.  However, since
the merger receives each flow record as its own filtered group, each merged
group has only one member. As a result, the ungrouper ends up rewriting the
merged groups as \texttt{flowtools} files using the same \texttt{zlib}
compression level. This means that the execution engine end up taking twice
the amount of time than the merger. It is important to note, that SiLK does
not have such an equivalent ungrouping tool and is therefore not considered in
this final evaluation.
\label{sec:evaluation}
\section{Conclusion}
We presented
\ac{NFQL}\footnote{\texttt{\url{https://github.com/vbajpai/nfql}}}, an
efficient C implementation of the stream-based flow query language. The
language allows applying absolute (or relative) filters, aggregating flows
into groups, evaluating timing relationships among them, and merging them into
one collection. \ac{NFQL} can execute such complex queries in matter of
minutes, thereby expanding the scope of current flow record processing tools.
The conducted performance evaluations reveal that \ac{NFQL} is on par with
tools that support only absolute filters. SiLK, the only openly available
package that provides tools that are similar to the rest of \ac{NFQL}'s
processing pipeline appears faster. This is because it can optimize its
operations in favor of the limited set of comparisons that are only based on
equality, and its usage of a different file storage format. The evaluation
queries developed as a part of this research work may also develop into a more
general benchmark suite for flow query tools and platforms.
\label{sec:conclusion}
%----------------------------------------------------------------------



% bibliography
%----------------------------------------------------------------------
\bibliographystyle{IEEEtran}
\bibliography{index}
%----------------------------------------------------------------------

\end{document}
