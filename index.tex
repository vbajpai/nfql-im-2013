\documentclass[conference]{IEEEtran}

% IEEEtran package options
%----------------------------------------------------------------------
% *** CITATION PACKAGES ***
\usepackage{cite}

% *** GRAPHICS RELATED PACKAGES ***
\usepackage[pdftex]{graphicx}
\graphicspath{{figures/}}
\DeclareGraphicsExtensions{.pdf,.eps}

% *** MATH PACKAGES ***
%\usepackage[cmex10]{amsmath}
%\interdisplaylinepenalty=2500

% *** SPECIALIZED LIST PACKAGES ***
%\usepackage{algorithmic}

% *** ALIGNMENT PACKAGES ***
%\usepackage{array}
%\usepackage{mdwmath}
%\usepackage{mdwtab}
%\usepackage{eqparbox}

% *** SUBFIGURE PACKAGES ***
%\usepackage[tight,footnotesize]{subfigure}
%\usepackage[caption=false,font=footnotesize]{subfig}

% *** FLOAT PACKAGES ***
%\usepackage{fixltx2e}
%\usepackage{stfloats}

% *** PDF, URL AND HYPERLINK PACKAGES ***
%\usepackage{url}
%----------------------------------------------------------------------



% added by Vaibhav
%----------------------------------------------------------------------
\usepackage[utf8]{inputenc}
\usepackage[nolist]{acronym}

\usepackage{scrtime}
\usepackage{prelim2e}
\usepackage{todonotes}
\renewcommand{\PrelimWords}{\relax}
\renewcommand{\PrelimText}{\footnotesize[\,\today\ at \thistime\,]}

\renewcommand{\ttdefault}{cmtt}
%----------------------------------------------------------------------

\begin{document}

\title{NFQL: A Swiss-Army Knife of Efficient Flow-Record Processing}

\author{\IEEEauthorblockN{Vaibhav Bajpai, Johannes Schauer, Jürgen
Schönwälder} \IEEEauthorblockA{School of Electrical and Computer Science\\
  Campus Ring 1, Jacobs University Bremen\\ \texttt{\{v.bajpai, j.schauer,
  j.schoenwaelder\}@jacobs-university.de} }}

\maketitle


\begin{abstract} Cisco's NetFlow protocol and IETF's IPFIX open standard have
  contributed heavily in pushing IP flow export as the de-facto technique for
  sending traffic patterns. These patterns have the potential to be used for
  billing and mediation, bandwidth provisioning, detecting malicious attacks
  and network performance evaluation. However, understanding these patterns
  requires sophisticated flow analysis tools that can mine them for such a
  usage. We recently proposed a framework design that can cap such
  flow-records to their full potential. In this paper, we introduce \ac{NFQL}.
  a holistic approach to an efficient implementation of the design. \ac{NFQL}
  can process flow records, aggregate them into groups, apply absolute (or
  relative) filters, invoke Allen interval algebra rules, and merge group
  records in matter of minutes. The implementation has been underpinned by
  suite of exhaustive benchmarks against contemporary flow-processing
  tools.\end{abstract}
% no keywords



\section{Introduction}

\section{Design}

\section{Implementation}
  \subsection{Filter}
  \subsection{Grouper}
  \subsubsection{Group Aggregations}
  \subsection{Group Filter}
  \subsection{Merger}
  \subsection{Ungrouper}

\section{Performance Evaluation}
\section{Related Work}
\section{Conclusion}
The \ac{NFQL} conclusion goes here \cite{vmarinov:2009}

% bibliography
%----------------------------------------------------------------------
% trigger a \newpage just before the given reference
% number - used to balance the columns on the last page
% adjust value as needed - may need to be readjusted if
% the document is modified later
%\IEEEtriggeratref{8}
% The "triggered" command can be changed if desired:
%\IEEEtriggercmd{\enlargethispage{-5in}}

\bibliographystyle{IEEEtran}
\bibliography{index}
%----------------------------------------------------------------------

\begin{acronym}[NFQL]
  \acro{NFQL}{Network Flow Query Language}
\end{acronym}

\end{document}

% IEEEtran figures, subfigures, tables
%----------------------------------------------------------------------
%\begin{figure}[!t]
%\centering
%\includegraphics[width=2.5in]{myfigure}
%\caption{Simulation Results}
%\label{fig_sim}
%\end{figure}

% (The subfig.sty package must be loaded for this to work.)
%\begin{figure*}[!t]
%\centerline{\subfloat[Case I]\includegraphics[width=2.5in]{subfigcase1}%
%\label{fig_first_case}}
%\hfil
%\subfloat[Case II]{\includegraphics[width=2.5in]{subfigcase2}%
%\label{fig_second_case}}}
%\caption{Simulation results}
%\label{fig_sim}
%\end{figure*}


%\begin{table}[!t]
%% increase table row spacing, adjust to taste
%\renewcommand{\arraystretch}{1.3}
% if using array.sty, it might be a good idea to tweak the value of
% \extrarowheight as needed to properly center the text within the cells
%\caption{An Example of a Table}
%\label{table_example}
%\centering
%% Some packages, such as MDW tools, offer better commands for making tables
%% than the plain LaTeX2e tabular which is used here.
%\begin{tabular}{|c||c|}
%\hline
%One & Two\\
%\hline
%Three & Four\\
%\hline
%\end{tabular}
%\end{table}
%----------------------------------------------------------------------


